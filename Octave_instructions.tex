\documentclass[11pt,letterpaper]{article}
\linespread{1}
\pagestyle{empty}
\usepackage{amsmath}
\usepackage[pdftex]{graphicx}
\usepackage{multicol}
\usepackage[letterpaper,left=0.75in,top=0.75in,right=0.75in,bottom=0.75in]{geometry}
\setlength{\parindent}{0pt}
\usepackage{enumerate}
  \setlength{\itemsep}{1pt}
\usepackage{caption}
%\usepackage[colorlinks=true]{hyperref}
\usepackage{hyperref}
\hypersetup{
 colorlinks=true,
 urlcolor=blue}
\usepackage{fancyhdr}
\pagestyle{fancy}
\fancyfoot[L]{CH262 Course Outline}
\fancyfoot[R]{Winter 2021}
\renewcommand{\footrulewidth}{0.4pt}
\renewcommand{\headrulewidth}{0pt}

\setlength{\parskip}{2pt}

\usepackage[scaled]{helvet}
\renewcommand\familydefault{\sfdefault} 
\usepackage[T1]{fontenc}

\begin{document}

\special{papersize=8.5in,11in}

\section*{Octave Instructions}

Octave is a \textit{high level programming language}.  This does not mean it is hard, or extremely advanced, or anything like that. Just that most of the details are taken care of ``\textit{under the hood}'' and  you can just focus on the math you are trying to solve instead of worrying about things like memory allocation and such.

I will be posting website versions of Octave code.  You can change the variables and rerun the code to see the influence on the output results. As I post them, I will demonstrate each Octave file during lecture time.

I am using the Jupyter environment to interact with Octave because using Jupyter I can also put in text that describes what the code is doing.  I host this Jupyter/Octave code on a website called Binder. This is what allows  you to run and interact with the code in your web browser.  This is a free service but sometimes there can be delays.

So if you find the website tedious you can always download the original Octave code and run it in Octave on your own computer.  I will post both the Octave links on MyLS as well as the original source code for Octave.  These are called mfiles.

I am not going to post the Jupyter code though.  Getting Jupyter up and running and ''\textit{talking}'' to the Octave ``\textit{kernel}'' is not always very easy and there is no real advantage to that.  All the information you need is in the Octave code anyway.  The Jupyter part is really just cosmetic.

To install Octave please go to this website (only if you are interested; this is not at all required)

\url{https://www.gnu.org/software/octave/index}





\end{document}
